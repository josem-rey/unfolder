\documentclass[11pt,a4paper,twoside,openright]{book}

\newcommand{\unfolder}{{\textsf{unfolder~}}}

\title{\unfolder: User Manual}
\author{Jos{\'e} M. Rey Poza}

\begin{document}

\maketitle

\tableofcontents

\chapter{What is \unfolder$\!\!$?}
\label{chap_what}


\unfolder is an experimental toolkit that enables the user to unfold
full fledged functional programs (i.e. programs that use higher order,
function composition and lazy evaluation with partially undefined
terms).

\unfolder is written in Prolog (more specifically, Ciao Prolog). In
its current state the programs to be unfolded are read from the Prolog
code itself, where they are represented as Prolog facts. These facts
represent guarded rules belonging to a functional program written in a
generic functional language. That is, \unfolder does not
currently read functional programs (written in Haskell or any other
existing functional language); instead, it reads the source to be
unfolded from an internal representation.

\section{Unfolding}

Unfolding is the process of replacing a function invocation by the
definition of that very same function. By repeating the process as
many times as necessary, what we get is a program that has the same
meaning as the original {\em folded} but that is simpler than the original
one in terms of function composition (a completely unfolded program
has no function composition at all but this is not possible in
general). That simpler form of the program can be seen as a set of
facts that plainly express the meaning of the original program (its
semantics).   

Unfolding generates valid functional code that can be run just as the
original one to produce the same result (provided full unfolding has
been possible).

Unfolding is an iterative process: Starting with an empty code, every
iteration produces a better approximation to the meaning of the
original code. These successive approximations draw closer to the real
meaning of the program by having more rules that deal with more input
values or by letting existing rules handle better approximations of
large terms.

Every approximation to the final meaning of the functional program
under examination is nothing more than a set of functional rules. The
set of functional rules contained within a given approximation is
called {\bf an interpretation}. The unfolding process begins with an
empty interpretation (denoted as $I_0$) and every iteration of
unfolding takes current interpreation along with the original program
rules to generate the next interpretation (that is, jumping from $I_n$
to $I_{n+1}$). 

\chapter{Writing Programs for \unfolder}
\label{chap_writing_programs}

As stated in chapter \ref{chap_what}, \unfolder reads the programs it
must unfold from Prolog source code. Programs to be unfolded are made
up of a number of rules. All rules must adhere to the following
format:

\begin{center}
{\tt rule(Function\_Name,Patterns,Guard,Body,Where,RuleId)}
\end{center}

\noindent
where each argument of {\tt rule} describes a part of a given
rule. For example, the two rules for the well known {\tt append}
function are written as follows:

\begin{verbatim}
rule(append,[nil,X],_,X,[]).
rule(append,[cons(X,Xs),Ys],_,cons(X,append(Xs,Ys)),[]).
\end{verbatim} 

\noindent
These two Prolog facts denote the following two Haskell rules:

\begin{verbatim}
append [] x = x
append (x:xs) ys = (x:(append xs ys))
\end{verbatim}

\noindent
The example above shows how program rules must be written:

\begin{itemize}
\item
Free functional variables are represented by free logic variables.

\item
Constructors are written as Prolog atoms (that is, they start by a
lowecase letter). Lists are represented by constractors {\tt nil} (for
{\tt []}) and {\tt cons} (for {\tt (\_:\_)}).

\item
No type declaration is necessary. The unfolder does not currently know
anything about types, so it is up to the user to ensure that all programs
are well typed.
\end{itemize}

\noindent
In turn, each part of the rule must obey the following restrictions:

\begin{itemize}
\item
The head must be an atom representing the name of the function. All
rules defining the same function must bear the same atom at this
position.

\item
The second argument for {\tt rule} represents the pattern of every
rule. A pattern is written as a list of terms. All functions defined
are considered to be {\em fully currified}. If tuples have to be used,
the can be written as \verb.(.$\mathit{term}_1,\ldots,\mathit{term}_n$
\verb.)..

\item
The guard of the rule is next. An empty guard is represented by a free
Prolog variable. If the guard is not empty, it must be written using
normal applicative notations (that is, $f(1,2)$ must be written as
{\tt f(1,2)}). Boolean conjunction is written as {\tt and(\_,\_)}.

\item
The fourth argument of {\tt rule} contains the body of the rule. It
has to be written according the same rules that govern the writing of
guards.

\item
The next argument is an experimental one intended to contain
local ({\em where}) declarations. It is currently unused and must
contain an empty list (\verb.[].).

\item
Finally, the last argument is an optional rule identifier that can be
used to let \unfolder track what rules have been used to generate
every fact. If a rule has to have an identifier, this argument must
hold such an identifier. The format to be used is {\tt [rule(<ID>)]}
where {\tt <ID>} is a Prolog atom or string containing the rule name. 
\end{itemize}

\chapter{Running \unfolder}
\label{chap_running}

\section{Execution}

\unfolder is currently included in one Prolog file that must be run
from within Ciao Prolog. Therefore, Ciao must be started and then {\tt
  unfolding.pl} must be loaded:

\begin{verbatim}
$ ciao
Ciao 1.15.0-14760: mié jun 27 18:36:00 CEST 2012 [install]
?- [unfolding].

yes
?-
\end{verbatim}

The rules that represent the functional program to be unfolded must,
at the time of writing this, be present inside the Prolog code.

As said before, unfolding is an iterative process. Basically, the
process of unfolding a functional program involves running an
iteration and examining its result until needed.

In \unfolder, an iteration is run by invoking the Prolog predicate \\
{\tt unfolding\_operator/0}. This takes the functional program
rules (represented as Prolog facts) and current interpretation to
generate the next interpretation.

Once {\tt unfolding\_operator/0} has been run, the content of current
interpretation can be examined by means of {\tt show\_int/0}.

The unfolding of the {\tt append} function is shown below:

\begin{verbatim}
$ ciao
Ciao 1.15.0-14760: mié jun 27 18:36:00 CEST 2012 [install]
?- [unfolding].

yes
?- unfolding_operator,show_int.
* append(Nil,b) = b
* append(Cons(b,c),d) = Cons(b,Bot)

yes
?- unfolding_operator,show_int.
* append(Nil,b) = b
* append(Cons(b,Nil),c) = Cons(b,c)
* append(Cons(b,Cons(c,d)),e) = Cons(b,Cons(c,Bot))

yes
?- unfolding_operator,show_int.
* append(Nil,b) = b
* append(Cons(b,Nil),c) = Cons(b,c)
* append(Cons(b,Cons(c,Nil)),d) = Cons(b,Cons(c,d))
* append(Cons(b,Cons(c,Cons(d,e))),f) = Cons(b,Cons(c,Cons(d,Bot)))

yes
?-
\end{verbatim}  

\noindent
Three unfolding iterations are shown above ($I_1, I_2$ and $I_3$). Note
that every step involves invoking {\tt unfolding\_operator/0} to
create a new interpretation and invoking {\tt show\_int/0} just after
that to see what the new interpretation contains. Of course, there is
no problem in invoking {\tt unfolding\_operator/0} any number of times
between two invocations of {\tt show\_int/0}.

\section{Understanding the Output of \unfolder}
\label{sec_understanding_output}

The output of \unfolder is the one generated by {\tt
  show\_int/0}. This predicate dumps current interpretation to the
screen. Every functional rule that belongs to current interpretation
is written in a somewhat beautified form. Although the result of
unfolding is valid functional code, \unfolder does not generate valid
Haskell code yet (in particular the rules are still uncurried). This
is work for the future.

The {\em beautification} of every interpretation element (every
generated functional rule) comprises the following steps:

\begin{itemize}
\item
Every interpretation element is preceded by an asterisk (\verb.'*'.)
to ease the reading of rules.

\item
Constructors are shown with uppercase initials. Lists are shown as
{\tt Nil} (for {\tt []}) and {\tt Cons} (for {\tt (\_:\_)}). Note that
{\tt Bot} ({\em bottom}) is an special constructor denoting the absence of
information (usally written as $\bot$ in books).

\item
Variables are shown with lowercase initials.

\item
Partial or higher order applications are represented by \\ 
{\em $<$function$>$}{\tt @}{\em $<$arguments\_list$>$} where {\em
  $<$function$>$} can be a free variable or the name of a function and
{\em $<$arguments\_list$>$} is a list of arguments already applied to the
function. 
\end{itemize}  

\noindent
The three iterations run above show how the unfolding process moves
closer and closer to the final meaning of the program until the final
full meaning is eventually found (which does not happen in this case
since {\tt append} has an infinite semantics). Every iteration
includes the previous one and adds more elements to it. Those new
elements are able to deal with successively longer lists.

The next example shows that unfolding does not always add more rules
to an interpretation even if the final result has not been
found. Consider the following code:

\begin{verbatim}
data Nat = Zero | Suc Nat

ones :: [Int]
ones = (1:ones)

take_n :: Nat -> [Int] -> Int
take_n Zero _ = []
take_n Suc(n) (x:xs) = (x:(take_n n xs))

main5 :: [Int]
main5 = take_n Suc(Suc(Zero)) ones
\end{verbatim}

\noindent
Its first unfolding iterations are as follows:

\begin{verbatim}
$ ciao
Ciao 1.15.0-14760: mié jun 27 18:36:00 CEST 2012 [install]
?- [unfolding].

yes
?- unfolding_operator,show_int.
* ones = Cons(1,Bot)
* take_n(Zero,b) = Nil
* take_n(Suc(b),Cons(c,d)) = Cons(c,Bot)

yes
?- unfolding_operator,show_int.
* take_n(Zero,b) = Nil
* ones = Cons(1,Cons(1,Bot))
* take_n(Suc(Zero),Cons(b,c)) = Cons(b,Nil)
* take_n(Suc(Suc(b)),Cons(c,Cons(d,e))) = Cons(c,Cons(d,Bot))
* main5 = Cons(1,Bot)

yes
?- unfolding_operator,show_int.
* take_n(Zero,b) = Nil
* take_n(Suc(Zero),Cons(b,c)) = Cons(b,Nil)
* ones = Cons(1,Cons(1,Cons(1,Bot)))
* take_n(Suc(Suc(Zero)),Cons(b,Cons(c,d))) = Cons(b,Cons(c,Nil))
* take_n(Suc(Suc(Suc(b))),Cons(c,Cons(d,Cons(e,f)))) = 
    Cons(c,Cons(d,Cons(e,Bot)))
* main5 = Cons(1,Cons(1,Bot))

yes
?- unfolding_operator,show_int.
* take_n(Zero,b) = Nil
* take_n(Suc(Zero),Cons(b,c)) = Cons(b,Nil)
* take_n(Suc(Suc(Zero)),Cons(b,Cons(c,d))) = Cons(b,Cons(c,Nil))
* ones = Cons(1,Cons(1,Cons(1,Cons(1,Bot))))
* take_n(Suc(Suc(Suc(Zero))),Cons(b,Cons(c,Cons(d,e)))) = 
    Cons(b,Cons(c,Cons(d,Nil)))
* take_n(Suc(Suc(Suc(Suc(b)))),Cons(c,Cons(d,Cons(e,Cons(f,g))))) = 
    Cons(c,Cons(d,Cons(e,Cons(f,Bot))))
* main5 = Cons(1,Cons(1,Nil))
\end{verbatim}

\noindent
Take a look at function {\tt ones}: It has just one rule in every
iteration. By contrast, {\tt take\_n} gathers more rules as iterations
go by. This is due to the fact that {\tt ones} has just one pattern
(the empty one) so all the rules are covered by a single case. Observe
how the value returned by that rule grows bigger with every iteration
(keep in mind that {\tt Bot} is the infimum -- least value -- of the
domain inside which interpretations are generated).

\section{Other Examples}

The code of \unfolder contains many additional examples covering the
main features of functional programming. They all are written as
Prolog rules belonging to the {\tt rule/5} predicate. All rules must
be commented out except for those that are about to be tested. The
steps for a test are the following ones:

\begin{enumerate}
\item
Uncomment the rules to be tested. Comment all the others.

\item
Load {\tt unfolding.pl} into Ciao Prolog.

\item
Run the iterations as described above.
\end{enumerate}

\noindent
The examples included in {\tt unfolding.pl} usually contain a {\tt
  mainN} function where {\tt N} is a number. This function can be seen
as the one that generates the answer that is being sought with the
particular test case.

Topics covered by the examples include:

\begin{itemize}
\item
Functions that need the {\tt match} operator (see the relavent paper).

\item
Guarded rules.

\item
Experimenting with constraints within guards.

\item
Dealing with infinite structures and lazy evaluation.

\item
Higher order and partial applications.

\item
Treatment of predefined functions (i.e. those with no rules).

\item
Function composition.

\end{itemize}

\chapter{Cleaning the Interpretations}

It was mentioned in the last Example of Section
\ref{sec_understanding_output} that some functions (like {\tt ones}) generate
successive facts that overlap, so the {\em smaller}, older facts are
made redundant by newer facts.

\unfolder contains code that {\em cleans} interpretations. A portion
of this code works automatically while some other portion has to be
manually invoked by the user.

The code that works automatically removes facts that are completelly
overlapped by existing facts. This was the case with function {\tt
  ones}. However, this behaviour is not always valid and can cause
loss of information with some programs. The programs that tend to
cause these problems are those containing functions that are only
partially defined or have rules that can never generate a fact.

This is why this automatic cleaning capability can be disabled and
replaced by some other methods that must be explicitly invoked by the
user. This way of working has been chosen because, although the manual
cleaning methods are valid for every program, they may generate
cumbersome interpretations (i.e. interpretations that are large and
whose facts contain large, complex guards).

As it is, \unfolder has its automatic cleaning code enabled. Every
interpretation is cleaned right after it is calculated. Such a
cleaning process is responsible for removing the old facts of {\tt
  ones} in Section \ref{sec_understanding_output}. 

On the opposite side, the code below shows when the automatic
behaviour is not desirable: 

\begin{verbatim}
data Nat = Zero | Suc Nat

f Zero = Zero

g x = Suc (f x)

h (Suc x) = Zero
\end{verbatim}

\noindent
Note that {\tt f} and {\tt h} are incomplete functions while {\tt g}
is complete. Programs with incomplete functions may arise this
incorrect behaviour of {\em clean}. In this case, \unfolder generates
the following interpretations ($I_0$ is empty, as usual):

\begin{verbatim}
Ciao 1.15.0-14760: mié jun 27 18:36:00 CEST 2012 [install]
?- [unfolding].

yes
?- unfolding_operator,show_int.
* f(Zero) = Zero
* g(b) = Suc(Bot)
* h(Suc(b)) = Suc(Zero)

yes
?- unfolding_operator,show_int.
* f(Zero) = Zero
* h(Suc(b)) = Suc(Zero)
* g(Zero) = Suc(Zero)
* main30 = Suc(Zero)

yes
?- 
\end{verbatim}

\noindent
The first set of facts belong to $I_1$ while the second set represents
$I_2$. Observe that {\tt g} is complete in $I_1$ but it is only defined for
{\tt Zero} in $I_2$: {\em clean} has removed information that was
essential for {\tt g}.

If the automatic cleaning functionality is disabled (which currently
requires a small code modification), $I_1$ is the same as before but
$I_2$ changes slightly:

\begin{verbatim}
* f(Zero) = Zero
* g(b) = Suc(Bot)
* h(Suc(b)) = Suc(Zero)
* g(Zero) = Suc(Zero)
* main30 = Suc(Zero)
\end{verbatim}

\noindent
Comparing this interpretation to the former $I_2$, we can see that
{\tt g} now has two fact that overlap: {\tt * g(b) = Suc(Bot)} and {\tt
  * g(Zero) = Suc(Zero)}. Such an overlapping must be solved but
neither fact can be removed completely or some information would be
lost.

The predicate {\tt reclean/1} can be called to see what facts
overlap. To see the overlappings that may exist inside the current
interpretation without modifying anything, it is called with a free
variable as argument:

\begin{verbatim}
?- reclean(X).

* f(Zero) = Zero

* g(b) = Suc(Bot)
*** g(Zero) = Suc(Zero)
* g(Zero) = Suc(Zero)

* h(Suc(b)) = Suc(Zero)

* main30 = Suc(Zero)
\end{verbatim}

\noindent
This shows, by means of indentation, that {\tt g(Zero) = Suc(Zero)} is
overlapped by {\tt * g(b) = Suc(Bot)}. To eliminate the overlapping
from the interpretation, the most general fact of the overlapping pair
must be modified so it can no longer be applied where the most
specific fact of the pair can. The predicate {\tt reclean/1} can also
perform that change at the user request. This is done by invoking {\tt
  reclean/1} with {\tt yes} as argument:

\begin{verbatim}
reclean(yes),show_int.
* f(Zero) = Zero
* h(Suc(b)) = Suc(Zero)
* g(Zero) = Suc(Zero)
* main30 = Suc(Zero)
* g(b) | Nunif([b],[Zero]) = Suc(Bot)

yes
?- 
\end{verbatim}   

\noindent
The operator {\tt Nunif/2} stands for {\em non-unification}. It
returns true if its two argument cannot be unified.

You can see that this latest interpretation does not have
overlappings: The last fact cannot be applied when its argument is
{\tt Zero}. At the same time, the other fact for {\tt g} (third fact)
cannot be applied to values other than {\tt Zero}. 

The invocation:

\begin{verbatim}
:- reclean(yes).
\end{verbatim}

\noindent
removes any overlapping that may exist inside the current
interpretation (that is, {\tt reclean} acts on all the functions of the
interpretation). 



\chapter{Outermost Unfolding}
\label{chap_outermost}

By default, \unfolder performs unfolding steps in all the possible
ways: That means that all the expressions that are headed by a
user-defined function are tested against the facts known so far in order
to check whether the pair $\langle$ {\em expression}, {\em fact}
$\rangle$ is suitable for unfolding.

This, of course, can generate a large number of redundant, overlapping
facts. These redundant facts will be removed by {\tt clean} and {\tt
 reclean} (or, alternatively, {\tt clean} and {\tt reclean} will
modify the existing facts so that they do not overlap).

However, if the generation of redundant facts is avoided as much as
possible, the necessity of removing or modifying facts is greatly
reduced and thus the system will run more efficiently.

It is well known that {\em outermost unfolding} prevents the evaluation of
expressions whose value is not needed to find the real value of a
larger expression (evaluation proceeds from the outermost position,
delving to innermost positions only if their value is strictly needed
to find the value of the larger expression).

\unfolder can simulate {\em outermost unfolding}. Its behaviour can be
changed from the default one (full unfolding of all suitable
positions) by using the following predicate:

\begin{verbatim}
:- assert_outermost_only.
\end{verbatim} 

\noindent
After the predicate above is executed all calls to {\tt
  unfolding\_operator/0} use {\em outermost unfolding}. The default
behaviour is recovered by executing the following predicate:

\begin{verbatim}
:- retract_outermost_only.
\end{verbatim} 

\section{Example: Differences Between Full and Outermost Unfolding}

Let us think of the following code:

\begin{verbatim}
data Nat = Zero | Suc Nat

leq :: Nat -> Nat -> Bool
leq Zero _ = True
leq (Suc _) Zero = False
leq (Suc x) (Suc y) = leq x y

g :: Nat -> Nat
g Zero = Zero

main22b :: Nat -> Bool
main22b x = leq Zero (g x)
\end{verbatim}

\noindent
This code translates into the following rules of \unfolder (including
names for all the rules):

\begin{verbatim}
rule(leq,[zero,_],_,true,[],[rule('L1')]).
rule(leq,[suc(_),zero],_,false,[],[rule('L2')]).
rule(leq,[suc(X),suc(Y)],_,leq(X,Y),[],[rule('L3')]).
rule(g,[zero],_,zero,[],[rule('G1')]).
rule(main22b,[X],_,leq(zero,g(X)),[],[rule('Main22b')]).
\end{verbatim}

\noindent
Using the unfolder's default behaviour, the following results are
obtained before cleaning (note that, since {\tt g} is incomplete, the
optimized version of {\tt clean} cannot be applied:

\begin{verbatim}
?- unfolding_operator,show_int.
* leq(Zero,b) = True  <L1>
* leq(Suc(b),Zero) = False  <L2>
* g(Zero) = Zero  <G1>

yes
?- unfolding_operator,show_int.
* leq(Zero,b) = True  <L1>
* leq(Suc(b),Zero) = False  <L2>
* g(Zero) = Zero  <G1>
* leq(Suc(Zero),Suc(b)) = True  <L3,L1>
* leq(Suc(Suc(b)),Suc(Zero)) = False  <L3,L2>
* main22b(b) = True  <Main22b,L1>
* main22b(Zero) = True  <Main22b,G1,L1>

yes
\end{verbatim}

\noindent
The execution above shows two interpretations ($I_1$ and $I_2$) for
the program given. Note that there are two overlapping facts for {\tt
  main22b}. Specifically, the second of those facts ({\tt *
  main22b(Zero) = True  <Main22b,G1,L1>}) owes its existence to the
the fact that the unfolding process has evaluated {\tt g x} (rule {\tt
  G1}) before evaluating {\tt leq} by means of rule {\tt L1}. Since
the value of {\tt g x} is irrelevant for {\tt main22b}, what that
superfluous rule usage has achieved is to create a fact that is not
needed (it is too specific).

{\em Outermost unfolding} would have avoided that superfluous
evaluation after realizing that the value of the inner expression {\tt
  g x} is not needed to find the value of {\tt main22b} since only
{\tt leq}'s first rule is usable to unfold the body of {\tt main22b}
and that rule does not demand the value of the position occupied by
{\tt g x}. This causes {\em outermost unfolding} to {\em ignore} that
position when unfolding. The result is as follows:

\begin{verbatim}
?- assert_outermost_only.

yes
?- unfolding_operator,show_int.           
* leq(Zero,b) = True  <L1>
* leq(Suc(b),Zero) = False  <L2>
* g(Zero) = Zero  <G1>

yes
?- unfolding_operator,show_int.

* leq(Zero,b) = True  <L1>
* leq(Suc(b),Zero) = False  <L2>
* g(Zero) = Zero  <G1>
* leq(Suc(Zero),Suc(b)) = True  <L3,L1>
* leq(Suc(Suc(b)),Suc(Zero)) = False  <L3,L2>
* main22b(b) = True  <Main22b,L1>

yes
\end{verbatim}

\noindent
Observe that both interpretations are identical to the ones before
except for the fact that was too specific: it is no longer there. This
has been caused because \unfolder has {\em purged out} the expression
{\tt g x} when considering the positions available for unfolding.

\end{document}
